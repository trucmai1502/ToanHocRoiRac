\chapter{Các Bài toán Đệ quy}\label{ch:1}

Ở chương này, chúng ta sẽ khám phá ba bài toán ví dụ.
Chúng có hai điểm chung:
đều đã được nghiên cứu rất kĩ bởi các nhà toán học;
và lời giải của chúng đều sử dụng ý tưởng \textit{đệ quy}, tức là lời giải cho mỗi bài toán phụ thuộc vào lời giải các bài toán con nhỏ hơn của bài toán đó.

\section{Bài toán Tháp Hà Tây}\label{sec:1.1}

\marginpar[Giơ tay lên nếu đây là lần đầu bạn nghe về bài toán này. Ok, số còn lại có thể lướt đến \eqref{eq:1.1}]{Giơ tay nếu đây là lần đầu bạn nghe về bài toán này. Ok, số còn lại có thể lướt đến \eqref{eq:1.1}}

Chúng ta sẽ xem xét một câu đố thú vị, tên là Bài toán Tháp Hà Tây, được phát minh bởi nhà toán học người Pháp Edouard Lucas năm $1883$.
Chúng ta có một tháp gồm tám đĩa, được xếp theo thứ tự kích thước giảm dần trên một trong ba cột.

\begin{center}
    \includegraphics[width=.5\textwidth]{assets/chapter1/Tower of Hanoi}
\end{center}

Mục tiêu của chúng ta, là di chuyển hết tòa tháp sang một trong các cột còn lại, trong đó mỗi lượt chỉ được di chuyển một đĩa từ cột này sang cột khác, và đĩa lớn không bao giờ được đặt trên đĩa nhỏ hơn.

Lucas còn viết một huyền thoại về tòa tháp Brahma khổng lồ, với 64 đĩa làm bằng vàng nguyên chất và ba cột làm từ kim cương.
\marginpar[Làm bằng vàng cơ à. Sao không phải bê tông?]{Làm bằng vàng cơ à. Sao không phải bê tông?}
Ông kể rằng, ngày mà thời gian bắt đầu trôi, Đấng Sáng Thế đã đặt những chiếc đĩa vàng này trên cột thứ nhất, rồi lệnh cho những nhà sư phải chuyển hết số đĩa sang cột thứ ba, theo quy tắc như trên.
Họ làm việc vất vả xuyên cả ngày đêm.
Khoảnh khắc họ đặt chiếc đĩa cuối cùng xuống, tòa Tháp sẽ sụp đổ, và thế giới sẽ đi đến hồi kết.

Không rõ ràng ngay lập tức rằng bài toán này có lời giải, nhưng khi quan sát kỹ hơn (hoặc biết trước bài toán này) sẽ thuyết phục chúng ta rằng nó có. Giờ câu hỏi chính được đưa ra: Chúng ta có thể làm tốt đến đâu? Ta cần ít nhất bao nhiêu lần di chuyển để hoàn thành được bài toán?

Một trong những hướng giải quyết tốt nhất cho những bài toán dạng trên: phương pháp tổng quát hóa. Tòa tháp Brahma có $64$ đĩa còn tòa tháp Hà Tây có $8$ đĩa; vậy ta sẽ tìm hiểu xem chuyện gì sẽ xảy ra nếu ta có $n$ chiếc đĩa.

Một lợi thế của phương pháp này là ta có thể thu nhỏ bài toán hơn nữa. Thực ra, qua cả cuốn sách, ta sẽ thấy sự hiệu quả của việc XÉT NHỮNG TRƯỜNG HỢP NHỎ HƠN trước. Khá dễ dàng để nhận ra cách để di chuyển tòa tháp với 1 hoặc 2 đĩa. Và chỉ cần thử nghiệm nho nhỏ sẽ cho ta thấy được cách tối ưu để giải bài toán với 3 đĩa.

Bước tiếp theo trong việc giải quyết vấn đề: đưa ra các ký hiệu phù hợp: ĐẶT TÊN VÀ CHINH PHỤC. Giả sử $\mathrm{T}_n$ là số lần di chuyển ít nhất để dịch chuyển $n$ cái đĩa từ cột này sang cột khác mà tuân theo luật của Lucas. Vậy hiển nhiên $\mathrm{T}_1 = 1$ và $\mathrm{T}_2 = 3$.

Ta cũng có thể moi được thêm một số thông tin bằng cách xét đến trường hợp nhỏ nhất: Đương nhiên $\mathrm{T}_0 = 0$ vì không bước di chuyển nào cần để chuyển tháp có $0$ đĩa! Những nhà toán học thông thái không bao giờ sợ phải nghĩ quá nhỏ, bởi vì những quy luật tổng quát sẽ dễ nhận thấy hơn khi các trường hợp đặc biệt đã được phân tích kỹ (kể cả khi những trường hợp đấy hiển nhiên).

Nhưng giờ hãy thay đổi cách nhìn của chúng ta và nhìn xa rộng hơn; làm thế nào để di chuyển tòa tháp với nhiều đĩa? Những thử nghiệm với $3$ đĩa chứng tỏ rằng để tối ưu số cách di chuyển, ta sẽ dịch chuyển $2$ đĩa đầu sang cột ở giữa, rồi di chuyển đĩa thứ ba và dịch chuyển $2$ đĩa còn lại. Những nhận xét này gợi ý cho chúng ta cách để dịch chuyển $n$ đĩa một cách tối ưu nhất: Đầu tiên, ta sẽ dịch chuyển $n - 1$ đĩa nhỏ nhất sang cột ở giữa (cần $\mathrm{T}_{n - 1}$ bước), rồi di chuyển đĩa lớn nhất (cần $1$ bước) sang cột thứ ba, và cuối cùng dịch chuyển $n - 1$ đĩa còn lại (cần thêm $\mathrm{T}_{n - 1}$ bước). Như vậy, ta sẽ chuyển được $n$ đĩa (với $n > 0$) qua nhiều nhất $2 \times \mathrm{T}_{n - 1} + 1$ bước:
$$\mathrm{T}_n \le 2 \times \mathrm{T}_{n - 1} + 1, \text{ \ \ \ \ với } n > 0.$$
Công thức trên sử dụng dấu '$\le$' thay vì dấu '$=$' bởi vì cách xử lý của chúng ta chỉ chứng minh $2 \times \mathrm{T}_{n - 1} + 1$ bước là đủ, nhưng ta vẫn chưa chứng minh tại sao $2 \times \mathrm{T}_{n - 1} + 1$ bước là cần để giải quyết bài toán. (Độc giả nào thông minh có thể nghĩ ra cách chứng minh trực tiếp công thức).

\marginpar[Phần lớn các "lời giải" được xuất bản để giải bài toán của Lucas, như lời giải của Allardice và Fraser\href{R. E. Allardice and A. Y. Fraser, La Tour d'Hanoi," Proceedings of the 2. Edinburgh Mathematical Society 2 (1884), 50 - 53}{$^{[7]}$}, đều không chứng minh vì sao $\mathrm{T}_{n} \ge 2 \times \mathrm{T}_{n - 1} + 1$]{Phần lớn các "lời giải" được xuất bản để giải bài toán của Lucas, như lời giải của Allardice và Fraser \href{R. E. Allardice and A. Y. Fraser, La Tour d'Hanoi," Proceedings of the 2. Edinburgh Mathematical Society 2 (1884), 50 - 53}{$^{[7]}$}, đều không chứng minh vì sao $\mathrm{T}_{n} \ge 2 \times \mathrm{T}_{n - 1} + 1$}

Nhưng ta có thể tìm cách nào tốt hơn không? Thực ra là không. Đến một lúc nào đó ta sẽ phải di chuyển cái đĩa ở đáy. Khi ta làm thế, $n - 1$ đĩa đầu phải ở trên phải ở cùng một cột nào đó, và nó sẽ mất ít nhất $\mathrm{T}_{n - 1}$. Nếu không cảnh giác, ta có thể di chuyển đĩa lớn nhất nhiều hơn 1 lần. Nhưng sau khi di chuyển đĩa lớn nhất lần cuối cùng, ta cần dịch chuyển lại tất cả $n - 1$ đĩa còn lại (nhớ là tất cả đĩa này phải trên cùng 1 cột) lên trên thằng lớn nhất; bước này cũng cần ít nhất $\mathrm{T}_{n - 1}$ bước. Vì thế, nên:
$$\mathrm{T}_n \ge 2 \times \mathrm{T}_{n - 1} + 1, \text{ \ \ \ \ với } n > 0.$$
Hai bất đẳng thức này, cùng với đáp án hiển nhiên với $n = 0$, cho ta:
\begin{equation}\label{eq:1.1}
    \begin{aligned}
        & \mathrm{T}_0 = 0; \\
        & \mathrm{T}_n = 2 \times \mathrm{T}_{n - 1} + 1, \text{ \ \ \ \ với } n > 0.
    \end{aligned}
\end{equation}
(Để ý rằng công thức trên đúng với những giá trị ta đã biết như $\mathrm{T}_1 = 1$ hay $\mathrm{T}_2 = 3$. Kinh nghiệm của chúng ta với những trường hợp nhỏ lẻ không những giúp chúng ta trong việc tìm ra công thức tổng quát, mà nó còn là công cụ tiện lợi để kiểm tra xem ta có mắc phải sai lầm chết người nào không. Những công cụ kiểm tra như thế này rất đáng giá khi ta đi sâu vào những kỹ thuật phức tạp hơn ở chương sau.)

Những tập hợp các biểu thức như \eqref{eq:1.1} được gọi là
\marginpar[Ừ, hình như mình nhìn thấy từ này rồi]{Ừ, hình như mình nhìn thấy từ này rồi}
công thức truy hồi. Chúng cho ta các giá trị tại biên và những biểu thức để tính giá trị tổng quát thông qua các giá trị trước. Đôi khi ta chỉ gọi một mình biểu thức là công thức truy hồi, dù cho thực tế nó cần giá trị biên để trở thành một công thức truy hồi. 

Công thức truy hồi có thể giúp chúng ta tính được $\mathrm{T}_n$ với mọi $n$ ta muốn. Nhưng không ai thich tính toán bằng công thức truy hồi cả;  đặc biệt khi $n$ khá lớn, nó sẽ tốn rất nhiều thời gian. Công thức truy hồi cũng chỉ thể hiện những thông tin gián tiếp và địa phương, không có tính tổng quát. Một thứ có thể giúp ta tránh khỏi những vấn đề này chính là \textit{giải pháp cho công thức truy hồi}. Đó chính là một công thức đẹp, gọn, còn gọi là "dạng đóng" của $\mathrm{T}_n$ để chúng ta có thể tính toán nhanh, ngay cả cho giá trị lớn của $n$. Với "dạng đóng" của $\mathrm{T}_n$, ta sẽ hiểu được bản chất của $\mathrm{T}_n$ là gì.

Vậy làm thế nào để chúng ta giải được công thức truy hồi này? Một cách ta có thể dùng là đoán đáp án, sau đó chứng minh rằng đáp án của chúng ta đúng. Và hy vọng lớn nhất của ta trong việc đoán đáp án chính là xét các trường hợp nhỏ của công thức truy hồi. Vậy ta tính toán được lần lượt, $\mathrm{T}_3 = 2 \times 3 + 1 = 7$, $\mathrm{T}_4 = 2 \times 7 + 1 = 15$, $\mathrm{T}_5 = 2 \times 15 + 1 = 31$, $\mathrm{T}_6 = 2 \times 31 + 1 = 63$. Ô! Nhìn như nó theo công thức:
\begin{equation}\label{eq:1.2}
    \begin{aligned}
        & \mathrm{T}_n = 2^n - 1, \text{ \ \ \ \ với } n > 0.
    \end{aligned}
\end{equation}
Ít nhất nó đúng với $n \le 6$.


\textit{Quy nạp} 
\marginpar[Quy nạp chứng minh rằng ta có thể trèo lên cao đến tùy thích cứ như chúng ta đang trên một chiếc thang, bằng cách chứng minh rằng ta có thể trèo lên bậc thấp nhất (cơ sở) và từ đó chứng minh với mỗi bậc tiếp theo ta có thể leo đến bặc tiếp theo nữa (quy nạp)]{Quy nạp chứng minh rằng ta có thể trèo lên cao đến tùy thích cứ như chúng ta đang trên một chiếc thang, bằng cách chứng minh rằng ta có thể trèo lên bậc thấp nhất (cơ sở) và từ đó chứng minh với mỗi bậc tiếp theo ta có thể leo đến bặc tiếp theo nữa (quy nạp)} 
là một trong những cách tổng quát để chứng minh rằng một nhận định nào đó về số nguyên $n$ đúng với mọi $n \ge n_0$. Đầu tiên ta chứng minh nhận định đúng khi n đặt giá trị nhỏ nhất, $n_0$, đây được gọi là bước cơ sở. Tiếp theo, ta chứng minh nhận định đúng với $n > n_0$, giả sử nhận định đã được chứng minh đúng với mọi giá trị từ $n_0$ đến $n - 1$; đây được gọi là bước quy nạp. Dạng chứng minh như thế này cho ta vô tận kết quả chỉ trong hữu hạn bước làm.



Các dạng bài giải công thức truy hồi rất lý tưởng cho cách chứng minh bằng quy nạp. Ở ví dụ của ta, ta có thể suy luận ra \eqref{eq:1.2} khá dễ dàng từ \eqref{eq:1.1}: Bước cơ sở hiển nhiên đúng, do $\mathrm{T}_0 = 2^0 - 1 = 0$. Và qua bước quy nạp, ta tiếp tục chứng minh khẳng định đúng với $n > 0$ nếu ta giả sử \eqref{eq:1.2} đúng khi n được thay thế bởi $n - 1$:
$$\mathrm{T}_n = 2 \times \mathrm{T}_{n - 1}  + 1 = 2 \times (2^{n - 1} - 1) + 1 = 2^n - 1.$$
Vậy \eqref{eq:1.2} đúng với $n$. Tuyệt! Hành trình đi tìm $\mathrm{T}_n$ của chúng ta đã thành công mỹ mãn.

Tất nhiên công việc của các nhà sư vẫn chưa kết thúc; họ vẫn đang phải miệt mài di chuyển những cái đĩa, và sẽ di chuyển những cái đĩa này khá lâu, do với $n = 64$ thì ta sẽ có $2^64 - 1$ bước (tầm khoảng 18 tỷ tỷ bước). Kể cả với tốc độ thần kỳ $1$ bước mỗi nano giây, họ sẽ phải tốn hơn 5 thiên niên kỷ chỉ để dịch chuyển tòa tháp Brahma. Bài toán gốc của Lucas dễ thở hơn một chút. Nó chỉ cần $2 ^ 8 - 1 = 255$ bước, sẽ mất khoảng 4 phút nếu nhanh tay.

Công thức truy hồi trong bài toán tháp Hà Tây khá điển hình trong những dạng có nhiều ứng dụng trong toán học. Khi đi tìm "dạng đóng" của một đại lượng nào đó ta quan tâm như $\mathrm{T}_n$, ta cần đi qua 3 bước:
\begin{enumerate}
    \item Xét các trường hợp bé trước. Làm thế giúp chúng ta có cái nhìn sâu sắc hơn về vấn đề và giúp ta ở bước 2 và 3.
    \item Tìm và chứng minh 
    \marginpar[Chứng minh là gì? "Một nửa của một phần trăm rượu nguyên chất"]{Chứng minh là gì? "Một nửa của một phần trăm rượu nguyên chất"}
    một biểu thức toán học cho đại lượng ta quan tâm. Với bài toán tháp Hà Tây, đó là công thức truy hồi \eqref{eq:1.1} giúp ta có thể tính được $\mathrm{T}_n$ với mọi $n$.
    \item Tìm và chứng minh "dạng đóng" của biểu thức của chúng ta. Với bài toán tháp Hà Tây, đó chính là \eqref{eq:1.2}
\end{enumerate}
Bước thứ ba sẽ là bước ta sẽ tập trung nhiều qua cả quyển sách. Thực ra, ta sẽ thường xuyên bỏ qua bước 1 và 2, vì ta sẽ có công thức truy hồi qua các biểu thức toán học ngay từ đầu. Nhưng ngay cả thế, ta sẽ làm rất nhiều các bài toán con mà sẽ đưa ta qua cả 3 bước.

Nghiên cứu của ta về tháp Hà Tây đã dẫn chúng ta đến với kết quả đúng, nhưng ta vẫn cần một "phép màu quy nạp"; ta vẫn phải dựa vào những dự đoán đầy may rủi của mình. Một trong những mục tiêu hàng đầu của quyển sách này chính là giải thích vì sao một người có thể giải quyết những công thức truy hồi mà không cần đến tâm linh. Ví dụ, ta có thể đơn giản hóa công thức truy hồi \eqref{eq:1.1} bằng cách cộng thêm $1$ vào cả hai vế của biểu thức:

\begin{equation*}
    \begin{aligned}
        & \mathrm{T}_0 + 1 = 1 \\
        & \mathrm{T}_n + 1 = 2 \times \mathrm{T}_{n - 1} + 2, \text{ \ \ \ \ với } n > 0. \\ 
    \end{aligned}
\end{equation*}
\marginpar[Thật thú vị! Ta ẩn đi hệ số $\mathbf{+1}$ trong \eqref{eq:1.1} bằng cộng thêm phần tử thay vì trừ nó đi]{Thật thú vị: ta ẩn đi hệ số $\mathbf{+1}$ trong \eqref{eq:1.1} bằng cộng thêm phần tử thay vì trừ nó đi}
Bây giờ nếu ta đặt $\mathrm{U}_n = \mathrm{T}_n + 1$, ta có:
\begin{equation}\label{eq:1.3}
    \begin{aligned}
        & \mathrm{U}_0 = 1 \\
        & \mathrm{U}_n = 2 \times \mathrm{U}_{n - 1}, \text{ \ \ \ \ với } n > 0.
    \end{aligned}
\end{equation}
Thực sự không cần là thiên tài cũng thấy được dạng tổng quát của công thức truy hồi này: $\mathrm{U}_n = 2^n$; do vậy $T_n = 2^n - 1$. Ngay cả máy tính cũng suy luận ra được điều đó.

\section{Đường thẳng trên mặt phẳng}\label{sec:1.2}

Bài toán ví dụ thứ hai của chúng ta mang hơi hướng hình học: Có bao nhiêu miếng bánh pizza một người có thể thu được qua n nhát cắt bởi dao cắt pizza? Hay nói cách khác: Số lượng khu vực $\mathrm{L}_n$ tối đa có thể tạo ra bởi n đường thẳng trên mặt phẳng? 
\marginpar[(Chiếc pizza với phô mai Thụy Sĩ?)]{(Chiếc pizza với phô mai Thụy Sĩ?)}
Bài toán này được giải vào năm 1826, bởi nhà toán học người Thụy Sĩ Jacob Steiner \href{J. Steiner, Einige Gesetze uber die Theilung der Ebene und des 5, 633. Raumes," Journal fur die reine und angewandte Mathematik 1 (1826), 349-364. Reprinted in his Gesammelte Werke, volume 1, 77-94}{$^{[338]}$}.

Một lần nữa ta bắt đầu bằng cách xét các những trường hợp nhỏ, luôn luôn bằng trường hợp nhỏ nhất trước. Mặt phẳng với 0 đường thẳng có 1 khu vực; với 1 đường thẳng có 2 khu vực; và với 2 đường thẳng ta có 4 khu vực:
\begin{center}
    \includegraphics[width=1\textwidth]{assets/chapter1/Line Sequence.png}
\end{center}
(Mỗi đường thẳng kéo dài vô tận ra hai phía.)

Thoạt nhìn ta thấy tất nhiên $\mathrm{L}_n = 2^n$! Vẽ thêm một đường thẳng nữa chỉ đơn giản là gấp đôi số khu vực chúng ta có. Thật đáng tiếc là chúng ta đã sai. Ta chỉ có thể gấp đôi chúng nếu đường thẳng thứ $n$ chia mỗi khu vực cũ thành hai khu vực mới khác nhau; và quả thật một khu vực cũ sẽ bị đường thẳng đấy cắt thành hai khu vực mới, do mọi khu vực cũ đều là hình lồi. 
\marginpar[Một khu vực được gọi là lồi khi và chỉ khi nó chứa tất cả các đoạn thẳng giữa hai điểm trong khu vực đó. (Đó không phải là định nghĩa trong từ điển của tôi, nhưng đó là quy ước của các nhà toán học với nhau.)]{Một khu vực được gọi là lồi khi và chỉ khi nó chứa tất cả các đoạn thẳng giữa hai điểm trong khu vực đó. (Đó không phải là định nghĩa trong từ điển của tôi, nhưng đó là quy ước của các nhà toán học với nhau.)}
Nhưng khi ta thêm đường thẳng thứ ba - đường thẳng được kẻ đậm ở hình dưới - ta nhận ra được rằng nó chỉ chia cắt nhiều nhất 3 khu vực cũ, không quan trọng ta đặt hai đường thẳng trước đó như thế nào:
\begin{center}
    \includegraphics[width=1\textwidth]{assets/chapter1/split_region.png}
\end{center}
Vậy $\mathrm{L}_3 = 4 + 3 = 7$ là số khu vực nhiều nhất ta có thể tạo được.

Sau khi suy nghĩ kỹ ta thấy được một cách tổng quát hóa phù hợp. Đường thẳng thứ $n$ (với $n > 0$) tăng số khu vực hiện tại lên $k$ lần khi và chỉ khi nó cắt qua $k$ khu vực, và nó cắt qua $k$ khu vực khi và chỉ khi đường thẳng đó cắt những đường thẳng trước tại $k - 1$ điểm. Hai đường thẳng cắt nhau tại tối đa 1 điểm. Vì vậy nên đường thẳng mới sẽ cắt $n - 1$ đường thẳng cũ tại tối đa $n - 1$ điểm khác nhau, và ta có điều kiện: $k \leq n$. Ta đã thiết lập được cận trên của đáp án
$$\mathrm{L}_n \leq \mathrm{L}_{n - 1} + n \text{, \ \ \ \ với } n > 0$$
Theo đó, khá là dễ để chứng minh dấu "=" ở công thức trên bằng quy nạp. Ta chỉ cần đặt đường thẳng thứ $n$ sao cho nó không song song với mọi đường thẳng khác (vậy nên nó cắt mọi đường thẳng), và sao cho nó không đi qua bất kỳ các điểm giao nhau nào trước đấy (thế nó mới cắt mọi đường thẳng tại các vị trí khác nhau). Vậy công thức truy hồi của bài này là
\begin{equation}\label{eq:1.4}
    \begin{aligned}
        & \mathrm{L}_0 = 1 \\
        & \mathrm{L}_n = \mathrm{L}_{n - 1} + n, \text{ \ \ \ \ với } n > 0.
    \end{aligned}
\end{equation}
Các giá trị $\mathrm{L}_1$, $\mathrm{L}_2$, $\mathrm{L}_3$ đều thỏa mãn công thức trên, nên có vẻ nó đúng.

Giờ ta cần tìm thêm "dạng đóng" của đáp án. Ta có thể chơi trò đoán, nhưng $1, 2, 4, 7, 16, \dots$ không có vẻ gì là nổi bật; vậy hãy chuyển sang một hướng khác. Ta có thể giải những công thức truy hồi này bằng cách "dải ra" từng số hạng một cho đến hết:
\marginpar["Dải ra"? Là tôi, tôi sẽ gọi đây là "thế vào"]{"Dải ra"? Là tôi, tôi sẽ gọi đây là "thế vào"}
\begin{equation*}
    \begin{split}
        \mathrm{L}_n & = \mathrm{L}_{n - 1} + n \\
        & = \mathrm{L}_{n - 2} + (n - 1) + n \\
        & = \mathrm{L}_{n - 3} + (n - 2) + (n - 1) + n \\
        & \ \ \ \ \ \ \ \ \ \ \ \vdots \\
        & = \mathrm{L}_0 + 1 + 2 + 3 + \dots + (n - 2) + (n - 1) + n \\
        & = 1 + \mathrm{S}_n, \ \ \ \ \text{với} \ \mathrm{S}_n = 1 + 2 + 3 + \dots + (n - 2) + (n - 1) + n
    \end{split}
\end{equation*}
Nói cách khác, $\mathrm{L}_{n}$ chỉ hơn 1 so với tổng $\mathrm{S}_n$ của $n$ số nguyên dương đầu tiên.

Đại lượng $\mathrm{S}_n$ xuất hiện khá nhiều, vì thế ta cần lập bảng cho những trường hợp nhỏ của đại lượng trên, như thế ta có thể nhận ra các số này dễ dàng hơn:

\begin{center}
    \begin{tabular}{c | *{14}{c}}
        $n$ & 1 & 2 & 3 & 4 & 5 & 6 & 7 & 8 & 9 & 10 & 11 & 12 & 13 & 14 \\
        \hline
        $\mathrm{S}_n$ & 1 & 3 & 6 & 10 & 15 & 21 & 28 & 36 & 45 & 55 & 66 & 78 & 91 & 105 \\ 
    \end{tabular}
\end{center}
Những giá trị trên còn được gọi là \textit{số tam giác}, bởi vì $\mathrm{S}_n$ là số con ki có trên một hình tam giác $n$ hàng. Ví dụ, một tam giác 4 hàng điển hình trong bowling có $\mathrm{S}_4 = 10$ con ki.

Để tính được giá trị $S_n$ ta có thể sử dụng một kỹ thuật mà Gauss 
\marginpar[Có vẻ như rất nhiều thứ được gán cho Gauss - hoặc ông cực kỳ thông minh hoặc ông có một người quản lý quảng cáo cực kỳ giỏi]{Có vẻ như rất nhiều thứ được gán cho Gauss - hoặc ông cực kỳ thông minh hoặc ông có một người quản lý quảng cáo cực kỳ giỏi} 
"đầu tiên" nghĩ ra vào năm 1786, khi ông mới 9 tuổi \href{G. Waldo Dunnington, Carl Friedrich Gauss: Titan of Science. Exposition Press, New York, 1955.}{$[88]$} (xem thêm \href{Leonhard Euler, Vollstandige Anleitung zur Algebra. Erster Theil. Von den verschiedenen Rechnungs-Arten, Verhaltnissen und Proportionen. St. Petersburg, 1770. Reprinted in his Opera Omnia, series 1, volume 1. Translated into Russian, 1768; Dutch, 1773; French, 1774; Latin, 1790; English, 1797.}{Euler [114, phần 1, §415]}):

\begin{center}
    \begin{tabular}{{r} * {10}{c}}
        $\mathrm{S}_n$ & = & 1 & + & 2 & + & $\dots$ & + & $(n - 1)$ & + & $n$ \\
        + $\mathrm{S}_n$ & = & $n$ & + & $(n - 1)$ & + & \dots & + & 2 & + & 1 \\
        \hline
        $2\mathrm{S}_n$ & = & $(n + 1)$ & + & $(n + 1)$ & + & \dots & + & $(n + 1)$ & + & $(n + 1)$
    \end{tabular}
\end{center}

\marginpar[Có lẽ ông ấy có bản tính làm mê hoặc lòng người]{Có lẽ ông ấy có bản tính làm mê hoặc lòng người}
Ta chỉ cần cộng $\mathrm{S}_n$ với đảo ngược của nó, sao cho mỗi cột trong $n$ cột bên phải cộng lại bằng $n + 1$. Rút gọn lại ta có được:
\begin{equation}\label{eq:1.5}
    \begin{aligned}
        & \mathrm{S}_n = \frac{n(n + 1)}{2}, \text{ \ \ \ \ với } n \geq 0.
    \end{aligned}
\end{equation}
Vậy ta đã có lời giải cho bài toán:
\begin{equation}\label{eq:1.6}
    \begin{aligned}
        & \mathrm{L}_n = \frac{n(n + 1)}{2} + 1, \text{ \ \ \ \ với } n \geq 0.
    \end{aligned}
\end{equation}
\marginpar[Thực ra Gauss được coi là một trong những nhà toán học vĩ đại nhất mọi thời đại. Vì vậy sẽ tốt khi ta hiểu được một trong những phát kiến của ông ta]{Thực ra Gauss được coi là một trong những nhà toán học vĩ đại nhất mọi thời đại. Vì vậy sẽ tốt khi ta hiểu được một trong những phát kiến của ông ta}

Là những chuyên gia, ta có thể khá mãn nguyện với bước biến đổi này và coi nó đã được chứng minh, dù cho ta bỏ qua chứng minh chặt chẽ khi ta "dải" các số hạng của $\mathrm{L}_n$ hay khi lật ngược $\mathrm{S}_n$ để tìm được công thức của nó. Nhưng những học trò của toán học như chúng ta phải đáp ứng được những tiêu chuẩn chặt hơn nữa, vì vậy sẽ rất tốt nếu ta chứng minh chặt chẽ công thức trên bằng quy nạp. Bước quy nạp quan trọng ở đây là:
$$\mathrm{L}_n = \mathrm{L}_{n - 1} + n = \left(\frac{1}{2}n(n - 1) + 1\right) + n = \frac{1}{2}n(n + 1) + 1$$
Giờ không còn nghi ngờ gì về "dạng đóng" của \eqref{eq:1.6} nữa.

\marginpar[Khi nghi ngờ, hãy để ý đến các từ ngữ. Tại sao nó lại là "đóng" mà không phải là "mở"? Những hình ảnh gì hiện lên khi ta nghĩ về những từ đấy?]{Khi nghi ngờ, hãy để ý các từ ngữ. Tại sao nó lại là "đóng" mà không phải là "mở"? Những hình ảnh gì hiện lên khi ta nghĩ về những từ đấy?}
Kỳ lạ thay ta liên tục nói về các "dạng đóng" mà không nói rõ ý của chúng ta là gì. Bình thường thì khá là rõ. Những công thức truy hồi như \eqref{eq:1.1} hay \eqref{eq:1.4} không phải là "dạng đóng" - chúng miêu tả một đại lượng bằng chính nó; nhưng những lời giải như \eqref{eq:1.2} hay \eqref{eq:1.6} có. Những tổng như $1 + 2 + \dots + n$ không phải "dạng đóng" - chúng ăn gian bằng cách dùng '$\dots$'; nhưng biểu thức như $\frac{n(n + 1)}{2}$ được coi là "dạng đóng". Ta có thể đưa
\marginpar[Câu trả lời: Biểu thức gọi là "đóng", không định nghĩa bằng chính nó - không tiếp tục thành một quan hệ truy hồi. Vụ án đã được "đóng" lại - nó sẽ không xảy ra thêm lần nữa. Ẩn dụ là chía khóa.]{Câu trả lời: Biểu thức gọi là "đóng", không định nghĩa bằng chính nó - không tiếp tục thành một quan hệ truy hồi. Vụ án đã được "đóng" lại - nó sẽ không xảy ra thêm lần nữa. Ẩn dụ là chía khóa.}
ra một định nghĩa hơi sơ sài như thế này: Một biểu thức cho đại lượng $f(n)$ được coi là "dạng đóng" khi ta có thể tính được chúng thông qua không quá một lượng xác định những phép tính "phổ biến", tiêu chuẩn. Lấy ví dụ, $2^n - 1$ và $\frac{(n + 1)n}{2}$ là dạng đóng bởi vì chúng chỉ dùng phép cộng, phép trừ, phép nhân, phép chia, và phép mũ một cách rõ ràng.

Tổng số lượng các "dạng đóng" đơn giản là hữu hạn, và có những công thức truy hồi mà không có "dạng đóng" nào đơn giản. Khi những công thức truy hồi đấy trở nên quan trọng, bởi vì nó xuất hiện liên tục, ta thêm một số phép tính vào trong kho vũ khí toán học của chúng ta; làm thế này sẽ giúp mở rộng đáng kể phạm vi các bài toán mà ta giải được bằng những dạng đóng "đơn giản". Ví dụ như tích của $n$ số tự nhiên đầu tiên, $n!$, đóng góp một phần quan trọng trong toán học mà ta coi nó như một phép tính cơ bản. Do đó công thức $n!$ được coi là "dạng đóng" trong khi dạng tương đương của nó $1 \times 2 \times \dots \times \ n$ không phải.

Giờ ta sẽ chuyển sang một biển thể của bài toán đường thẳng trên mặt phẳng: Giả sử thay vì ta dùng đường thẳng ta dùng đường gấp khúc, mỗi đường tạo bởi 2 tia. Vậy thì số lượng khu vực $\mathrm{Z}_n$ lớn nhất ta tạo ra được bằng n đường gấp khúc như vậy là bao nhiêu? Ta có thể đoán $\mathrm{Z}_n$ sẽ gấp đôi hoặc gấp ba lần $\mathrm{L}_n$. Hãy kiểm tra xem:

\begin{center}
    \includegraphics[width=1\textwidth]{assets/chapter1/small-case-zig-zag.png}
\end{center}

Từ hai trường hợp nhỏ trên, cùng với một chút suy nghĩ,
\marginpar[... và với một chút suy nghĩ lại ...]{... và với một chút suy nghĩ lại ...}
ta nhận ra rằng đường gấp khúc của chúng ta chính là hai đường thẳng trừ những khu vực được gộp lại khi mà hai đường thẳng đó không "vượt" quá điểm giao giữa bọn chúng.
\begin{center}
    \includegraphics[width=1\textwidth]{assets/chapter1/zig-zag-explanation.png}
\end{center}
Khu vực 2, 3, và 4, thông thường sẽ được tách biệt bằng hai đường thẳng, trở thành 1 khu vực duy nhất khi có đường gấp khúc như vậy, và ta mất hai khu vực. Tuy vậy, nếu ta đặt mọi thứ chính xác - mọi điểm bị gấp khúc của các đường đều nằm ngoài các giao điểm của các đường đó - và đó là tất cả ta mất; hai khu vực với mỗi đường.
\marginpar[Bài tập 18 đi vào chi tiết hơn]{Bài tập 18 sẽ đi vào chi tiết hơn}
Do đó: 
\begin{equation}\label{eq:1.7}
    \begin{aligned}
        \mathrm{Z}_n = \mathrm{L}_{2n} - 2n & = 2n(2n + 1) / 2 + 1 - 2n \\
                                            & = 2n^2 - n + 1, \text{\ \ \ \ với } n \geq 0
    \end{aligned}
\end{equation}
So sánh các dạng đóng như \eqref{eq:1.6} với \eqref{eq:1.7}, ta thấy được rằng với giá trị lớn của $n$,
\begin{equation*}
    \begin{aligned}
        \mathrm{L}_n & \sim \frac{1}{2}n^2, \\
        \mathrm{Z}_n & \sim 2n^2; \\
    \end{aligned}
\end{equation*}
vì vậy dùng đường gấp khúc của ta sẽ tạo ra số khu vực gấp 4 lần so với số khu vực tạo ra được bởi đường thẳng. (Trong những chương sau ta sẽ đề cập đến làm thế nào để phân tích các tính chất khi xấp xỉ các hàm nguyên với $n$ lớn. Kí hiệu '$\sim$' được định nghĩa tại phần 
9.1% \ref{sec:9.1}
).